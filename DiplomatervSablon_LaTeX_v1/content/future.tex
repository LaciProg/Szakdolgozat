%----------------------------------------------------------------------------
\chapter{Továbbfejlesztési lehetőségek}
\label{sec:Future}
%----------------------------------------------------------------------------

Az alkalmazást számos területen lehet még fejleszteni.  
Rengeteg kihasználatlan lehetőség rejlik még benne.  
Ebben a fejezetben szeretném bemutatni mindazt, ami a fejlesztés közben eszembe jutott, de ebbe a dolgozatba tartalmilag nem fért bele.

A megjelenítés tekintetében sokat lehetne még fejleszteni.  
Lehetőséget lehetne biztosítani a felhasználó számára, hogy testreszabja az alkalmazás színvilágát és megjelenését.  
Több komponens is van, amelynek más megjelenést adva szebbé lehetne tenni az alkalmazást, mindezt úgy, hogy a funkcionalitás megmaradjon.

Az alkalmazást ki lehetne bővíteni más platformokra, például iOS eszközökre, illetve a webre is.  
A webes technológia jelenleg még elég bizonytalan, alpha verzióban van, de már most érdemes lehet elkezdeni kísérletezni vele.  
A másik népszerű operációs rendszerre, az iOS-re, sajnos csak Mac eszközön lehet fejleszteni, pontosabban a szoftvert lefordítani és egy emulátort vagy szimulátort futtatni.  
Ennek beszerzése vagy hozzáférése után érdekes lehetőség lehetne a Compose Multiplatform ezen részének a kipróbálása is.

Felhasználók és intézmények létrehozása biztosítaná, hogy egymástól elkülönítve lehessen az alkalmazást használni.  
Az intézményeken belül fel lehetne venni tárgyakat, amelyekhez a kérdések és feladatsorok tartoznak.  
Ennek megvalósítása egy fejlett autentikációs és regisztrációs rendszer létrehozását igényli, ami jelentős változtatást jelentene az adatbázis sémájában.  
Ez a felhasználói felületeken és a REST API-n valamivel kisebb módosítást igényelne, de az autentikáció megvalósítása nem triviális feladat egy ilyen környezetben.

A felhasználók bevezetése után bővíthető lenne tanulói és tanári szerepekkel.  
Az ellenőrző képernyőhöz hasonlóan létre lehetne hozni egy feladatsor-kitöltő felületet.  
A tanulók kitölthetnék ezeket, és erről kapnának egy statisztikát, amelyet össze lehetne vetni más korábbi kitöltésekkel.

Bővíthető lenne a kérdések típusa is.  
Például rövid válaszok esetén, ahol néhány helyes szó megfelelő, a válaszok automatikusan javíthatók lennének.  
A hosszabb válaszok, rajzolás vagy kódírás javítása nehezebb feladat.  
Egyrészt ezek felismerése nem triviális, sokkal fejlettebb szövegfelismerő megoldások szükségesek.  
Az ilyen válaszok automatikus javításához mesterséges intelligenciát kellene fejleszteni, amelyet a backend érhetne el a javítás céljából.

Mindent összevetve az alkalmazás tovább bővíthető mind vizuális megjelenés, mind funkcionalitás, mind pedig platformok terén.
