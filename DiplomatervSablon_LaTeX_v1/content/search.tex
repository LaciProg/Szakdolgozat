%----------------------------------------------------------------------------
\chapter{Irodalomkutatás}
\label{sec:Search}
%----------------------------------------------------------------------------

\section{Felhasznált technológiák}
\label{sec:Technologies}

Ebben a fejezetben bemutatom az általam használt technológiákat amiket használtam és segítettek ennek a dolgozatnak a megírásában és elkészítésében.

\subsection{Jetpack Compose}

A Jetpack Compose a korábbi Android fejlesztési módszer mellett hozott létre egy alternatív megoldást.
Kezedtebn nem lehetett tudni, hogy a fejlesztők hogyan fognak reagálni az új irányra.
Korábban a Java nyelv mellett megjelent a Kotlin nyelv is, ami később szinte teljesen leváltotta az elődjét.
Ebből arra lehetett következtetni, hogy egy új és modernebb megoldás meg tudja állni a helyét az XML View megoldást ellenében.
Jelenleg mind a két megoldás támogatott, de a fejlsztések iránya egyértelműen a Compose felé húz.

"A Jetpack Compose egy új, deklaratív UI toolkit, amit a Google hozott létre kifejezetten natív Android alkalmazások fejlesztéséhez."\cite{GettingStartedWithJetpackCompose}
A deklaratív nyelvekhez hasonlóan, azt kellmegadnunk, hogy mit szeretnénk látni és nem azt leírni, hogy ez hogyan történjen meg.
Nekünk elegendő azt leírni, hogy a gomb hogyan nézzen ki és hol legyen és megadni egy lambda paraméternek, hogy a megnyomása soránmi történjen.
Mivel ez egy UI toolkit, ezért az összes vezerlő és szerkezeti elem hasonlóan néz ki, hasonlóan lehet használni így a fejlesztői és a felhasználói élmény is egységes és megszokott minőségű lesz minden alkalommal.
A Google ezt a Material design segítségével hozta létre, illetve annak újabb változataival. Erről részletesebben itt találhatók információk: \url{https://m3.material.io/}.

Az alábbiakban egy a Google által írt rövid kódrészleten bemutatom a legfontosabb részeit a Compose alapjainak.\cite{BasicCodelab}
Az első lépése az alaklamzás elkészítésének az a Composable függvény megírása.
Minden a UI-t megjelenítő függvény a @Composable annotációt viseli.
Innentől kezdve hagyományos Kotlin függvényként viselkedik, megadahatunk tetszőleges paramétereket (name, modifier) és alapértelmezett értékeket is.
Egy Composable függvényből tetszőleges másik Composable függvény meghívható megfelelőláthatóság esetén.
Ilyen például a Text() is ami a Material könyvtárnak egyik tagja és egyszerű szöveget jelenít meg.

A UI megírása után ezt a megfelelő helyen meg is kell jelenítenünk, erre az alkalmazás belépési pontja után van lehetőségünk.
Android esetén ez az activity onCreate függvénye.
A setContent vár egy lambda függvényt, aminek viselnie kell a @Composable annotációt, használhatjuk hozzá a Kotlin trailing lambda megoldását, aholis, ha az utolsó paramétere a függvénynek egy lambda, akkor a többi paraméter megadása után {} között megadhatjuk afüggvény törzsét.
A BasicsCodelabTheme is egy Composable függvény amiben az alap beállítások után meghívhatjuk a saját Greeting függvényünket.
A UI felépítse innentől kezdve már egyszerű.
A Composable függvények megírása után egymásból meghívva azokat előáll az alkalmazás.

\begin{lstlisting}
    @Composable
    fun Greeting(name: String, modifier: Modifier = Modifier) {
        Text(
            text = "Hello $name!",
            modifier = modifier
        )
    }

    class MainActivity : AppCompatActivity() {
        override fun onCreate(savedInstanceState: Bundle?) {
            super.onCreate(savedInstanceState)
            setContent {
                BasicsCodelabTheme {
                    // A surface container using the 'background' color from the theme
                    Surface(
                    modifier = Modifier.fillMaxSize(),
                    color = MaterialTheme.colorScheme.background
                    ) {
                        Greeting("Android")
                    }
                }
            }
        }
    }

    // Jetpack Compose forráskód
    public fun ComponentActivity.setContent(
        parent: CompositionContext? = null,
        content: @Composable () -> Unit
    )
\end{lstlisting}

A következőkben bemutatom a UI toolkit fontosabb általam használt részeit.
Kitérek arra, hogy mire jó, miért ezt használtam és hogyan lehet őket hatékonyan felhasználni a legújabb Compose Multiplatform verziókban.

\subsection{MutableState és StateFlow}
\url{https://www.jetbrains.com/help/kotlin-multiplatform-dev/compose-viewmodel.html#using-viewmodel-in-common-code} 

\subsection{ViewModel}
\url{https://www.jetbrains.com/help/kotlin-multiplatform-dev/compose-viewmodel.html#using-viewmodel-in-common-code}

\subsubsection{Navigation and routing}
\url{https://www.jetbrains.com/help/kotlin-multiplatform-dev/compose-navigation-routing.html}


\subsection{KotlinX Serilizáció}

\subsection{Ktor}

\subsection{CameraX}

\subsection{ML-Kit}

\subsection{ACCOMPANIST-engedélykezelés}

\subsection{pdfbox és PdfDocument}

\section{Kotlin- és Compose Multiplatform}

\section{Gradle build rendszer}

\section{Fejlesztőkörnyeztek}

\section{REST API és adatbázis}

\section{Kipróbált, de végül nem használt egyéb érdekes megoldások}