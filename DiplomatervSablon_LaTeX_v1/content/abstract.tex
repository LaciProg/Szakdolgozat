\pagenumbering{roman}
\setcounter{page}{1}

\selecthungarian

%----------------------------------------------------------------------------
% Abstract in Hungarian
%----------------------------------------------------------------------------
\chapter*{Kivonat}\addcontentsline{toc}{chapter}{Kivonat}

A közoktatásban és a felsőoktatásban is gyakori probléma mind a tanárok, mind a diákok számára az időhiány a rengeteg munka miatt.
Ez az alkalmazás a tanárok munkáját hivatott segíteni az előző félévben elkészített REST API-t felhasználva, és a hozzá írt Android-eszközökre készült program továbbfejlesztése által.
Az alkalmazás lehetőséget biztosít egy nagyméretű kérdésbank létrehozására és tárolására.
A kérdések bármikor módosíthatók, törölhetők, vagy hozhatók létre újak.
Természetesen nem csak egy ember dolgozhat ugyanazon a tárgyon; a kérdésbank és a számonkérések közösen szerkeszthetők.

Az alkalmazás lehetőséget biztosít a kérdések létrehozása mellett egyéni témakörök létrehozására, amivel a kérdéseket és feladatsorokat lehet csoportosítani.
Továbbá el lehet készíteni a saját pontrendszert, akár több félét is, amelyet dinamikusan lehet változtatni a kérdéseknél.
Fontos szempont volt az automatizált javítás segítése, így csak egyszerű kérdések vannak: igaz-hamis és feleletválasztós kérdések.
Sajnos az AI még nem tart ott, hogy bármilyen kézírást pontosan felismerjen, és ebből a szövegből megállapítsa annak helyességét. Ennek ellenére a szövegfelismerő funkció így is támogatja a javítást, aminek az eredményét megkapja a javító.

Ezekből az elemekből áll össze a számonkérés.
Ez a szoftver csak a kérdéssorok összeállításáért és kiértékeléséért felel. 
Ennek megfelelően elő kell állítani magát a feladatsort. 
Egy dolgozatot ki lehet exportálni PDF formátumban, erről egy előnézet is lesz, amin nagyjából látszik, hogyan fog kinézni, de a végső változat csak az exportálást követően fog látszani. 
Ezt követően szabadon nyomtathatóvá válik.

Egy modern szoftver esetén elvárt, hogy könnyen és kényelmesen lehessen kezelni, mindenki számára a neki tetsző módon. 
Ennek alapján úgy döntöttem, hogy felhasználom az Android fejlesztésben szerzett tapasztalataimat. 
2021 augusztusában jelent meg a Compose Multiplatform technológia, amely kedvez az Android-fejlesztőknek, mivel a natív Android-megoldások könnyen átültethetők egy cross-platform alkalmazásba. 
Jelenleg stabilan működik Android-, asztali- és iOS-alkalmazások készítéséhez, eszköz hiányában az első kettőt készítettem el.


\vfill
\selectenglish


%----------------------------------------------------------------------------
% Abstract in English
%----------------------------------------------------------------------------
\chapter*{Abstract}\addcontentsline{toc}{chapter}{Abstract}

In both public and higher education, time constraints are a common issue for both teachers and students due to the large workload. 
This application is intended to assist teachers by utilizing the REST API developed in the previous semester and enhancing the program created for Android devices. 
The application allows for the creation and storage of a large question bank. 
Questions can be modified, deleted, or new ones can be created at any time. 
Of course, more than one person can work on the same subject; the question bank and the tests can be edited collaboratively.

In addition to creating questions, the application also allows for the creation of custom topics, which can be used to organize questions and assignments. 
Furthermore, a custom scoring system can be created, even multiple types, which can be dynamically adjusted for different questions. 
An important aspect was to assist in automated grading, so only simple questions are included: true/false and multiple-choice questions. 
Unfortunately, AI is not yet at the level where it can accurately recognize any handwriting and determine its correctness from the text. 
Nonetheless, the text recognition function still supports grading, and the results are provided to the grader.

These elements come together to form the assessment. 
This software is responsible solely for compiling and evaluating the question sheets. 
Accordingly, the task sheet itself must be generated. A test can be exported in PDF format, with a preview available that roughly shows how it will look, though the final version will only be visible after exporting. 
After this, it can be freely printed.

For modern software, it is expected to be easy and convenient to use, allowing everyone to handle it in their preferred way. 
Based on this, I decided to leverage my experience in Android development. 
The Compose Multiplatform technology, released in August 2021, is favorable for Android developers, as native Android solutions can easily be adapted into a cross-platform application. 
It currently works stably for creating Android, desktop, and iOS applications; due to a lack of devices, I have implemented the first two.


\vfill
\selectthesislanguage

\newcounter{romanPage}
\setcounter{romanPage}{\value{page}}
\stepcounter{romanPage}